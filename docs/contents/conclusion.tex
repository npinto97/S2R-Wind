This study explored the effectiveness of semi-supervised regression models for wind power forecasting under varying degrees of label scarcity. Using a realistic SCADA dataset with over 20,000 training instances per fold, we evaluated three modeling strategies: a co-training-based ensemble method (S2RMS), a decision tree-based semi-supervised learner (CLUS+), and two fully supervised baselines (ElasticNet and XGBoost). Experiments were performed across eight temporal folds and multiple labeled data proportions (10\%, 20\%, 70\%).

The results demonstrate that CLUS+ consistently delivers superior predictive performance among the semi-supervised methods. At just 20\% labeled data, CLUS+ matches or exceeds the accuracy of fully supervised baselines trained on the entire dataset. This level of efficiency is critical in operational settings where labeling is costly or infeasible at scale. Its ability to robustly integrate unlabeled instances, without requiring aggressive pseudo-labeling or iterative retraining, makes it particularly effective in high-dimensional, time-dependent domains such as wind power prediction.

In contrast, S2RMS exhibited unstable and subpar performance across all scales. Despite modifications to improve efficiency, such as restricting candidate sets and leveraging GPU acceleration, S2RMS failed to generalize reliably. Its high computational cost and memory usage further undermine its practicality. These shortcomings align with the original S2RMS design, which was validated on small-scale datasets of only 2,000 examples per run. When scaled to realistic data volumes, the method struggles to maintain performance and tractability.

CLUS+ also excels in terms of computational efficiency. It trains significantly faster than S2RMS while maintaining its predictive strength across a wide range of label budgets. The method’s native support for symbolic missing values, ensemble-based architecture, and minimal reliance on hyperparameter tuning make it an attractive candidate for real-world deployments.

Overall, this work confirms the practical viability of semi-supervised tree ensembles for large-scale wind power forecasting. In particular, CLUS+ emerges as a scalable, accurate, and resource-efficient solution, outperforming both traditional supervised baselines and modern SSL alternatives when evaluated under realistic constraints.
