The increasing availability of high-resolution sensor data in the energy sector has opened new opportunities for predictive modeling of renewable sources such as wind power. However, in many industrial settings, the cost of acquiring labeled data remains a bottleneck. Supervisory control and data acquisition (SCADA) systems collect abundant time series data, but the process of curating accurate labels, especially for forecasting or fault detection, is often expensive and limited in scope. This motivates the development of models that can effectively leverage partially labeled or weakly supervised datasets.

In this work, we explore the use of semi-supervised regression techniques to improve wind turbine power output prediction using limited labeled samples. We focus on a real-world dataset derived from the Baidu KDD Cup 2022 wind power forecasting challenge, consisting of SCADA time series data from multiple wind turbines over several months. The objective is to predict future power output given historical measurements and turbine metadata.

We evaluate the performance of a state-of-the-art semi-supervised method, S2RMS (Semi-Supervised Regression with Multiple Similarities)\cite{liu2024semi}, which integrates co-training, similarity-based sample selection, and triplet-based feature learning. The framework is compared against strong baselines including ElasticNet, XGBoost, and CLUS+\cite{petkovic2023clusplus}, a decision tree-based system that supports semi-supervised learning through Predictive Clustering Trees.

To ensure a robust evaluation, we employ a fold-based temporal split strategy, where training and test sets are generated from non-overlapping time windows. For each fold, experiments are conducted using varying proportions of labeled data (10\%, 20\%, 70\%), with performance measured across standard regression metrics such as RMSE, MAE, RSE, and R$^2$. Our methodology is designed to assess both the label-efficiency and generalization capability of each model in low-label regimes.

The goal of this study is to systematically benchmark semi-supervised approaches for wind power regression and to understand their behavior across different data availability scenarios.

