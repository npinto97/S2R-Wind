Wind power forecasting is a critical task for the operation and integration of renewable energy into the electrical grid. However, obtaining labeled data for training predictive models, such as actual power output (PATV) measurements, is expensive and limited in practice. In contrast, large volumes of unlabeled sensor data are typically available from wind turbines. This imbalance motivates the use of \textit{semi-supervised learning (SSL)} techniques that can leverage both labeled and unlabeled instances to improve performance.

This project investigates the application of semi-supervised regression methods on a realistic wind turbine dataset derived from the Baidu KDD Cup 2022\footnote{\href{https://aistudio.baidu.com/competition/detail/152/0/introduction}{https://aistudio.baidu.com/competition/detail/152/0/introduction}}. Our aim is to evaluate how effectively SSL models can improve forecasting performance when labeled data is scarce, and to benchmark these models against strong supervised baselines.

We focus primarily on \textbf{S2RMS}, a co-training-based ensemble SSL framework developed in Python, and compare it against a state-of-the-art tree-based SSL method implemented in \textbf{CLUS+}, a decision tree learner supporting predictive clustering trees (PCTs) with semi-supervised extensions. We further include \textbf{ElasticNet} and \textbf{XGBoost} as supervised baselines, enabling a rigorous comparative evaluation.

To ensure fairness and reproducibility, all models are trained and evaluated across eight temporal folds with varying percentages (10\%, 20\%, 70\%) of labeled data. The training, evaluation, and data generation processes follow a standardized pipeline aligned with the CRISP-DM methodology. Performance is measured using RMSE, MAE, RSE, and R$^2$, with additional analyses on the influence of label availability and model robustness.

The remainder of this document describes the methodological pipeline, models, and results in detail, offering a data-driven perspective on the utility and limitations of semi-supervised learning in practical wind power prediction settings.
